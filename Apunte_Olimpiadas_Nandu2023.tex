\documentclass{article}

\usepackage{amsmath, amsthm, amssymb, amsfonts}
\usepackage{thmtools}
\usepackage{graphicx}
\usepackage{setspace}
\usepackage{geometry}
\usepackage{float}
\usepackage{hyperref}
\usepackage[utf8]{inputenc}
\usepackage[english]{babel}
\usepackage{framed}
\usepackage[dvipsnames]{xcolor}
\usepackage{tcolorbox}
\usepackage{tikz}

\graphicspath{ {./images/} }

\colorlet{LightGray}{White!90!Periwinkle}
\colorlet{LightOrange}{Orange!15}
\colorlet{LightGreen}{Green!15}

\newcommand{\HRule}[1]{\rule{\linewidth}{#1}}

\declaretheoremstyle[name=Ejemplo,]{thmsty}
\declaretheorem[style=thmsty,numberwithin=subsection]{ejemplo}

\declaretheoremstyle[name=Theorem,]{thmsty}
\declaretheorem[style=thmsty,numberwithin=section]{theorem}
\tcolorboxenvironment{theorem}{colback=LightGray}

\declaretheoremstyle[name=Proposition,]{prosty}
\declaretheorem[style=prosty,numberlike=theorem]{proposition}
\tcolorboxenvironment{proposition}{colback=LightOrange}

\declaretheoremstyle[name=Principle,]{prcpsty}
\declaretheorem[style=prcpsty,numberlike=theorem]{principle}
\tcolorboxenvironment{principle}{colback=LightGreen}

\setstretch{1.2}
\geometry{
    textheight=9in,
    textwidth=5.5in,
    top=1in,
    headheight=12pt,
    headsep=25pt,
    footskip=30pt
}

% ------------------------------------------------------------------------------

\begin{document}

% ------------------------------------------------------------------------------
% Cover Page and ToC
% ------------------------------------------------------------------------------

\title{ \normalsize \textsc{}
		\\ [2.0cm]
		\HRule{1.5pt} \\
		\LARGE \textbf{\uppercase{Apunte Olimpíadas Ñandú}
		\HRule{2.0pt} \\ [0.6cm] \vspace*{10\baselineskip}}
		}
\date{}
\author{\textbf{Autor} \\ 
		Ignacio Cuevas \\
		Córdoba, Argentina \\
		2023}

\maketitle
\newpage

\tableofcontents
\newpage

\section{Introducción a las Competencias}

\subparagraph*{}
\begin{normalsize}
En este apunte vamos a estudiar los diferentes tipos de problemas que nos podemos encontrar en la Olimpíada de Matemáticas Ñandú (Tercer nivel) y vamos a aprender algunas herramientas que nos serán útil para resolver cada uno de los problemas.
\end{normalsize}

\newpage

\section{La Competencia}

\subparagraph*{}
\begin{normalsize}
La competencia consta de 3 ejercicios:
\begin{enumerate}
	\item El primer ejercicio de Álgebra
	\item El segundo de Geometría
	\item Y el tercero de Conteo
\end{enumerate}
Ahora pasaremos a explicar cada uno de ellos.
\end{normalsize}

\subsection{Álgebra}
\begin{normalsize}
En este primer tipo de ejercicios nos vamos a encontrar problemas donde nos presentan un tipo de objeto (por ejemplo, verduras), y ciertos valores de ellos (por ejemplo, su precio), ya sea el objeto individual o sumas de varios. Veamos un ejemplo para que quede más claro.
\\
\end{normalsize}

\begin{ejemplo}
En las bolsas A, B y C hay en total 200 caramelos. En la bolsa B hay 20 más que
en la A, en la bolsa C hay 61 más que en la B. ¿Cuántos caramelos hay en la bolsa C?
\end{ejemplo}

\begin{ejemplo}
En el colegio hay 1360 alumnos inscriptos. De los alumnos inscriptos, $\frac{3}{5}$ se anotaron en el turno mañana. De los alumnos del turno mañana, $\frac{1}{4}$ van al jardín, $\frac{2}{3}$ van a la primaria y los demás van a la secundaria. ¿Cuántos alumnos van a la secundaria en el turno mañana?
\end{ejemplo}

\subsection{Geometría}
\begin{normalsize}
Este segundo ejercicio es de geometría, es decir, nos vamos a encontrar con ejercicios que involucran figuras, lados, ángulos.
El dibujo puede estar dado en el ejercicio, o pueden dar las características para que uno lo grafique. Estos dibujos muchas veces tienen puntos clave representados con letras.
Los datos que nos dan pueden ser las medidas de algunos lados, ángulos, perímetros, áreas, etc.
\\ \\
\end{normalsize}

\begin{ejemplo}
ABCD es un rectángulo,
AB = 3BC;
M es punto medio de AB;
N es punto medio de AD;
P es punto medio de CD;
O es el punto medio del segmento MP.
El perímetro de AMPD es de 80cm.
¿Cuál es el perímetro de AMON?
¿Cuál es el área de BCPO?

\includegraphics[scale=0.6]{geometry-example-1}
\\
\\
\end{ejemplo}

\begin{ejemplo}
En la figura:
ABC es un triángulo equilátero,
EFGH es un rectángulo,
CDE es un triángulo rectángulo,
F es punto medio de AB,
G es punto medio de BC,
DE = AB,
Perímetro de ABC = 96cm.
¿Cuál es el área de EFGH?
¿Cuál es el perímetro de CDFG?
¿Cuál es el área de ABCD?
¿Cuál es el área de CDG?

\includegraphics[scale=0.6]{geometry-example-2}
\end{ejemplo}

\subsection{Conteo}
\begin{normalsize}
Este último ejercicio, como bien dice su nombre, tenemos que contar cosas. Pueden ser números en un cierto rango que cumplen ciertas propiedades, formas de hacer determinada cosa, u otras cosas. Como datos nos van a dar el problema y sus restricciones.
\\
\\
\end{normalsize}

\begin{ejemplo}
Luis come, en total, 5 chocolatines en la semana, de lunes a viernes.
Puede comer 2 chocolatines, 1 chocolatín o ningún chocolatín en el mismo día.
En esta tabla anota cuántos chocolatines come cada día.
¿De cuántas maneras distintas puede completar la tabla?
Explica cómo las contaste.
\\
\\
\end{ejemplo}

\begin{ejemplo}
Aníbal y Beto están en el equipo de pingpong.
Martín, Nicolás, Oscar y Pablo están en el equipo de tenis.
Ramón, Santiago y Tomás están en el equipo de natación.
Entre estos deportistas deben elegir un grupo de 5 para hacer un viaje.
Si en el grupo debe haber por lo menos un representante de cada deporte,
¿de cuántas maneras distintas puede hacerse la elección?
Explica cómo las contaste.

\end{ejemplo}

\newpage

\section{Álgebra}
\begin{normalsize}
En este primer módulo vamos a ver que herramientas nos serán de gran utilidad para resolver ejercicios del primer tipo mencionado en la introducción.
Nos vamos a enfocar en:
\begin{enumerate}
	\item Manejar bien las unidades y los números.
	\item Entender la noción de ecuación.
	\item Apender a resolver ecuaciones.
	\item Interpretar un problema para armar ecuaciones.
\end{enumerate}
\end{normalsize}

\subsection{Manejo de números}
\begin{normalsize}
Antes de largarnos a aprender ecuaciones es importante que entendamos que hay diferentes formas de representar los números. Es decir que podemos ver dos numeros con representaciones distintas y que valen lo mismo.
\end{normalsize}

\subsubsection*{Números enteros y reales}
\begin{normalsize}
Los numeros enteros representan cosas, como bien dice su nombre, enteras. Podemos decir \textit{1, 2, 3 autos}, pero no podemos decir \textit{un auto y medio}, o \textit{3.14 autos}. Para esta otra representacion existen los números reales, por ejemplo podemos decir, \textit{1.5 litros de leche}, o \textit{11.86 cm}.
Cada uno de estos tipos tienen propiedades que no vamos a ver en esta sección, sino en la de \textit{Conteo}.
\end{normalsize}

\subsubsection*{Fracciones}
\begin{normalsize}
Las fracciones son formas de representar números reales con números enteros. Muchas veces se usan para llegar a un resultado más exacto o para no trabajar con números con decimales (estos pueden ser más difíciles de manipular). Un número fraccionario consta de dos partes, el numerador (arriba) y el denominador (abajo),
\[\frac{x}{y}=x:y=x/y\]
Por ejemplo, podemos leer la siguiente fracción de varias formas $\frac{3}{4}$, tres cuartos, tres dividido cuatro, tres sobre cuatro, tres partes de cuatro, 0.75, entre muchas otras formas.
\\
\end{normalsize}

\begin{normalsize}
\begin{center}
\textbf{Operaciones con fracciones}
\end{center}
\end{normalsize}

\begin{normalsize}
\textbf{Suma y resta:} Hay dos tipos de sumas, con mismo denominador y con distinto denominador.
Con mismo denominador se suman directamente los numeradores.
\[\frac{x}{d}+\frac{y}{d}=\frac{x+y}{d}\]
\[\frac{x}{d}-\frac{y}{d}=\frac{x-y}{d}\]

Con distinto denominador:
\begin{itemize}
	\item Denominador: se multiplican los denominadores.
	\item Numerador: Se multiplican numerador con denominador contrario y se suman.
\end{itemize}
Veamos la fórmula:
\[\frac{x}{a}+\frac{y}{b}=\frac{xb+ya}{ab}\]

\textbf{Producto:} se multiplica numerador con numerador, y denominador con denominador.
\[\frac{x}{a}\cdot\frac{y}{b}=\frac{xy}{ab}\]

\textbf{División:} en el numerador del resultado se coloca el producto entre el primer numerador con el denominador del segundo, y en el denominador del resultado el producto entre denominador del primero con el numerador del segundo.
\[\frac{x}{a}\cdot\frac{y}{b}=\frac{xb}{ay}\]\\

\textbf{Nota:} Cuando operamos entre fracciones y números enteros, es útil ver que un número entero tiene denominador 1. Por ejemplo, $4=\frac{4}{1}$\\

\textbf{Nota:} Si multiplicamos o dividimos dos numeros inversos, es decir que el numerador de uno es el denominador del otro y viceversa, el resultado es 1. Por ejemplo, $\frac{2}{5}\cdot\frac{5}{2}=1$, lo mismo con un entero, $4\cdot\frac{1}{4}=\frac{4}{1}\cdot\frac{1}{4}=1$\\

\textbf{Definición:} Decimos que el número inverso de $\frac{a}{b}$ es $\frac{b}{a}$\\

\textbf{Observación:} Podemos decir que dividir $\frac{a}{b}$ es lo mismo que multiplicar $a$ por el inverso de $b$. $\frac{a}{b}=a\cdot\frac{1}{b}$\\

\textbf{Observación:} Cuando tenemos $-\frac{a}{b}$ es lo mismo que decir $\frac{-a}{b}$, o bien, $\frac{a}{-b}$. Pues, $-(\frac{a}{b})=(-1)\frac{a}{b}=\frac{(-1)a}{b}=\frac{-a}{b}$\\
\end{normalsize}

\subsubsection*{Porcentajes}
\begin{normalsize}
Los porcentajes se aplican a una cantidad, por ejemplo \textit{el 50\% de las manzanas}, o \textit{se aumentó un 110\% del precio original}.
Cada porcentaje se puede representar como un número real.
Por ejemplo,
\begin{itemize}
	\item El 25\% de B, es lo mismo que $0.25B$, o $\frac{1}{4}B$
	\item El 110\% de C, es lo mismo que $1.1C$, o $\frac{11}{10}C$
	\item El 10\% menos de D, es lo mismo que $D-0.1D=0.9D$
\end{itemize}
\end{normalsize}

\subsubsection*{Propiedades importantes}
\begin{normalsize}
\begin{itemize}
	\item \textit{Asociativa:} $a+(b+c)=(a+b)+c$, vale tambien para el producto $a(bc)=(ab)c$.
	\item \textit{Permutativa:} $a+b=b+a$, vale tambien para el producto $ab=ba$.
	\item \textit{Distributiva:} $a(b+c)=ab+ac$, no vale al reves.
\end{itemize}

\textbf{Nota:} Es lo mismo restar que sumar el opuesto $a-b=a+(-b)$.
\end{normalsize}

\subsection{Noción de ecuacion}
\begin{normalsize}
\textbf{Definicion:} Una ecuación es una igualdad. Consta de tres partes, dos expresiones y un símbolo de igual ($=$) entre ellas.
\[expr1=expr2\]
Entonces, como tenemos que mantener siempre la igualdad, si sumamos $4$ de un lado, tenemos que sumar $4$ del otro.
\[expr1+4=expr2+4\]
Podemos ver un ejemplo concreto,
\[(3+2)=5\]
\[(3+2)+4=5+4\]
pero,
\[(3+2)+2\neq5+4\]
Lo mismo pasa con las demas operaciones, si restamos en un lado, restamos lo mismo en el otro, si multiplicamos o dividimos de un lado, hacemos lo mismo del otro.
\end{normalsize}

\subsection{Resolución de ecuaciones con una incógnita}
\begin{normalsize}
Que pasa cuando en una ecuación, que antes vimos que es una igualdad, hay una incógnita (o valor desconocido). Cómo podemos hacer para obtener su/sus valores posibles. Eso es lo que vamos a estar viendo en esta parte.

\textbf{Definición:} Una incógnita es un valor desconocido que buscamos encontrar. En general, esta se representa con letras. Por ejemplo, 
\[2x=4\]
\[6b+7=13\]
\[\frac{3+k}{2}=8\]
\[1.1m-\frac{2m}{3}=1.25\]
\\
Para resolver una ecuación con una incógnita tenemos que aplicar operaciones inteligentemente (recordemos que aplicamos la misma operación a ambos lados), para que nos quede nuestra incógnita sola (o despejada) en uno de los lados.
Un par de aclaraciones antes de ver un ejemplo.
\begin{itemize}
	\item Es lo mismo escribir $x$ que $1\cdot x$, es lo mismo escribir $2x$ que $2\cdot x$
	\item No podemos sumar $x+4$, simplemente dejamos la expresión como esta. Tenemos que sumar números con números, e incógnitas del mismo tipo. (Nota: sí podemos multiplicar).
	
	Por ejemplo, $5x+2x=7x$ y $2x+7+3x+1=5x+8$.\\
\end{itemize}

Ahora veamos un ejemplo con los pasos a seguir para resolver una ecuación con una incógnita,

\begin{align}
2x+7 &= 10\\2x+7-7&=10-7\\2x+0&=10-7\\2x&=3\\\frac{2x}{2}&=\frac{3}{2}\\x&=\frac{3}{2}
\end{align}

Y listo, ya obtuvimos el valor de $x=\frac{3}{2}$\\

Veamos esto paso por paso,
\begin{enumerate}
	\item Planteamos la ecuación.
	\item Como buscamos \textit{despejar x}, nos conviene restar \textit{7} en ambos lados.
	\item Resolvemos el lado izquierdo (recordemos las aclaraciones antes mencionadas).
	\item Resolvemos el lado derecho.
	\item Ahora nos conviene dividir ambos lados por \textit{2}, ya que $\frac{2}{2}=1$ y $1x=x$
	\item Finalmente resolvemos el lado izquierdo, ya que en el derecho no queda nada por resolver y vemos que nos queda la \textit{x despejada}.
\end{enumerate}
\end{normalsize}

\subsubsection*{Ejercicios de práctica}
\begin{normalsize}
\begin{enumerate}
	\item $6x-4=26$
	\item $12x+8=5x+36$
	\item $3(2x-6)=2(5x+3)$
	\item $\frac{4x+5}{5}=\frac{x+17}{3}$
\end{enumerate}
\end{normalsize}

\subsection{Cómo interpretar un problema}
\begin{normalsize}
\textbf{Importante:} Para interpretar y plantear un problema con facilidad y rapidez, es necesario de mucha práctica.

La resolución de problemas de matemáticas recorre cuatro fases:
\begin{enumerate}
	\item Comprender el problema
	\item Elaborar un plan para resolverlo
	\item Ejecutar el plan
	\item Comprobar que la respuesta es correcta
\end{enumerate}

Veamos varios ejemplos para que se entienda.
\\
\end{normalsize}

\begin{ejemplo}
El doble de un número es igual a 10.

\begin{enumerate}
	\item El problema es, encontrar un número tal que multiplicado por 2 es 10.
	\item Podemos plantear la siguiente ecuación: $2n=10$
	\item Resolvemos la ecuación.
		\begin{align}
		2n&=10\nonumber\\
		\frac{2n}{2}&=\frac{10}{2}\nonumber\\
		n&=5\nonumber
		\end{align}
	\item Comprobamos que este bien. $2\cdot5=10$
\end{enumerate}

\end{ejemplo}

\begin{ejemplo}
La mitad de un número más 5 es igual a 8.

\begin{enumerate}
	\item Acá es un poco diferente, pero sabemos que la mitad de un número es lo mismo que dividir un número por 2. El problema que nos plantean es, encontrar un número tal que, si lo dividimos por 2 ($\frac{k}{2}$) y a ese resultado le sumamos 5 ($\frac{k}{2}+5$), obtenemos 8 ($\frac{k}{2}+5=8$).
	\item Planteemos la ecuación: $\frac{k}{2}+5=8$
	\item Resolvamos para k.
	\begin{align}
	\frac{k}{2}+5&=8\nonumber\\
	\frac{k}{2}+5-5&=8-5\nonumber\\
	\frac{k}{2}&=3\nonumber\\
	\frac{k}{2}\cdot2&=3\cdot2\nonumber\\
	k\cdot1&=6\nonumber
	\end{align}
	\item Comprobamos que este bien. $\frac{6}{2}+5=3+5=8$
\end{enumerate}

\end{ejemplo}

\begin{ejemplo}
En el kiosco, 1 gaseosa cuesta \$12 y 1 jugo cuesta \$7.
Compré 2 gaseosas, 1 jugo y 3 paquetes de galletitas. Pagué \$49 en total.
¿Cuál es el precio de cada paquete de galletitas?

\begin{enumerate}
	\item Los datos que nos dan son los precios de algunos productos, y sabemos que la suma de cierta cantidad de productos da \$49. Nuestra incógnita en este caso es el precio de las galletitas.
	\item Antes de empezar veamos lo siguiente, llamemos G=gaseosa, J=jugo, M=galletitas \textit{(de masitas, no uso G de galletitas porque ya G significa gaseosa)}:
		\begin{itemize}
			\item $1\cdot G=G=12$, una gaseosa cuesta \$12.
			\item $1\cdot J=J=7$, un jugo \$7.
			\item $M=?$
		\end{itemize}
	Ahora si podemos plantear la ecuación. Tenemos que decir que 2 gaseosas, 1 jugo y 3 paquetes de galletitas cuestan \$49, entonces:
	\[2G+J+3M=49\]
	Pero algunos valores ya los tenemos,
	\[2\cdot12+7+3M=49\]
	\[24+7+3M=49\]
	\[31+3M=49\]
	\item Una vez que sustituimos los valores correspondientes, procedemos a resolver la ecuación:
		\begin{align}
		31+3M&=49\nonumber\\
		31+3M-31&=49-31\nonumber\\
		3M&=18\nonumber\\
		\frac{3M}{3}&=\frac{18}{3}\nonumber\\
		M&=6\nonumber
		\end{align}
	\item Veamos que se cumple lo primero que planteamos con el valor obtenido:
	\[2G+J+3M=49\]
	\[2\cdot12+7+3\cdot6=49\]
	\[24+7+18=49\]
	\[49=49\]
\end{enumerate}

\end{ejemplo}

\begin{ejemplo}
De los socios del club $\frac{7}{8}$ se anotaron para ir a la cena de fin de año, pero $\frac{1}{4}$ de los anotados no fueron a la cena. Si había 315 socios en la cena, ¿cuántos socios se anotaron para la cena?, ¿cuántos socios tiene el club?
\\ \\
En principio, este problema parece tener pocos datos, pero veremos que son suficientes.
\begin{enumerate}
	\item El primer dato que tenemos claro es que hay 315 socios en la cena. Ahora, ¿cuál es el valor desconocido? Es la cantidad de socios totales del club, pues con ese valor podemos calcular fácilmente la cantidad de anotados para la cena, ($\frac{7}{8}$ del total).
	
Llamemos $S$ a la cantidad total de socios que tiene el club. Sabemos que de \textbf{todos} los socios del club, se anotaron $\frac{7}{8}$ de ellos. Más formalmente $\frac{7}{8}S$. Y de esa cantidad, $\frac{1}{4}$ no fue a la cena, o dicho de otro modo, $\frac{3}{4}$ si fueron.

	\item Entonces, si ya \textit{"conocemos"} la porción de personas que fueron a la cena, podemos plantear la ecuación:
	\[(\frac{7}{8}S)(\frac{3}{4})=315\]
	\item Resolvamos para $S$.
	\begin{align}
	(\frac{7}{8}S)(\frac{3}{4})&=315\nonumber\\
	\frac{21}{32}S&=315\nonumber\\
	(\frac{21}{32}S)\cdot\frac{32}{21}&=315\cdot\frac{32}{21}\nonumber\\
	S&=\frac{10080}{21}\nonumber\\
	S&=480\nonumber
	\end{align}
	Ya conocemos el valor de $S$, que habiamos dicho que era el total de socios del club. Ahora nos falta ver cuántos socios se anotaron para la cena, que en un principio sabíamos que era $\frac{7}{8}S=\frac{7}{8}\cdot480=420$. Entonces la respuesta es: la cantidad total de socios que tiene el club son 480, de esos, 420 se anotaron para ir a la cena. Pregunta: ¿Cuántos no fueron a la cena?
	\item Comprobamos que esté bien. $(\frac{7}{8}S)(\frac{3}{4})=(\frac{7}{8}\cdot480)(\frac{3}{4})=420\cdot(\frac{3}{4})=\frac{1260}{4}=315$
\end{enumerate}

\end{ejemplo}

\subsection{Cómo resolver ecuaciones con más de una incógnita}
\begin{normalsize}
La resolucion de ecuaciones con mas de una incognita es bastante similar a resolver una con una incognita.
Pero antes de lanzarse a entender como resolverlas veamos algunas definiciones importantes:\\\\
\textbf{Definicion:} un \textit{sistema de ecuaciones} es un conjunto de ecuaciones con más de una incógnita que conforman un problema matemático que consiste en encontrar los valores de las incógnitas que satisfacen dichas operaciones.\\\\
\textbf{Definicion:} una \textit{solucion} es un conjunto de valores tal que si se reemplazan las incognitas del sistema por esos valores, se cumplen todas las ecuaciones simultaneamente.\\\\
\textbf{Definicion:} decimos que una incognita esta \textit{despejada} si esta en uno de los dos lados de la igualdad y es la unica vez que aparece en toda la ecuacion.\\\\
\textbf{Definicion:} dos expresiones son \textit{equivalentes} si valen lo mismo.\\\\
\textbf{Observacion:} podemos reemplazar expresiones equivalentes (iguales por iguales).\\\\
\textbf{Nota:} Para obtener una unica solucion, es necesario tener la misma cantidad de ecuaciones que de incognitas.
\\\\
\end{normalsize}

\subsubsection*{Metodos de resolucion de sistemas de ecuaciones}
\begin{normalsize}
Existen 3 metodos para resolver sistemas de ecuaciones:
\begin{enumerate}
	\item Sustitucion
	\item Igualacion
	\item Reduccion o eliminacion
\end{enumerate}
Nosotros nos vamos a centrar en el primero.\\\\
% Buscar un buen link que explique los otros metodos
Primero veamos un ejemplo para entender la idea general del \textit{metodo de sustitucion}.

\begin{ejemplo}
Dar la solucion del siguiente sistema de ecuaciones:
	\begin{equation}
		\begin{cases}
		3x-2y=5\nonumber\\
		-x+3y=3	
		\end{cases}
	\end{equation}

El metodo consiste en despejar una de las incognitas en la primer ecuacion, supongamos $x$, y reemplazamos su equivalencia en la segunda ecuacion.
Hagamos el primer paso:
	\begin{align}
	3x-2y&=5\nonumber\\
	3x&=5+2y\nonumber\\
	x&=\frac{5+2y}{3}\nonumber
	\end{align}
\end{ejemplo}

Como siempre mantuvimos la igualdad, sabemos que \textit{x es equivalente a $\frac{5+2y}{3}$}. Entonces es \textit{"legal"} reemplazar su equivalencia en la segunda ecuacion.
	\begin{align}
	-x+3y&=3\nonumber\\
	-(\frac{5+2y}{3})+3y&=3\nonumber
	\end{align}
La ecuacion que nos queda es una ecuacion con una sola incognita, y esto ya sabemos como resolverlo.
	\begin{align}
	-(\frac{5+2y}{3})+3y&=3\nonumber\\
	\frac{-5-2y}{3}+3y&=3\nonumber\\
	\frac{-5}{3}-\frac{-2y}{3}+3y&=3\nonumber\\
	\frac{-5}{3}-\frac{-2}{3}y+3y&=3\nonumber\\
	-\frac{-2}{3}y+3y&=3-\frac{-5}{3}\nonumber\\
	\frac{7}{3}y&=3-\frac{-5}{3}\nonumber\\
	\frac{7}{3}y&=3+\frac{5}{3}\nonumber\\
	\frac{7}{3}y&=\frac{14}{3}\nonumber\\
	7y&=\frac{14}{3}\cdot3\nonumber\\
	7y&=14\nonumber\\
	y&=\frac{14}{7}\nonumber\\
	y&=2\nonumber
	\end{align}
Pero todavia no terminamos, no sabemos aun cual es el valor de $x$. Lo unico que tenemos que hacer para conocer su valor es reemplazar el valor de $y=2$ en la ecuacion que habiamos despejado $x$, $x=\frac{5+2y}{3}$.
	\begin{align}
	x&=\frac{5+2y}{3}\nonumber\\
	x&=\frac{5+2\cdot2}{3}\nonumber\\
	x&=\frac{5+4}{3}\nonumber\\
	x&=\frac{9}{3}\nonumber\\
	x&=3\nonumber
	\end{align}
Por lo tanto, la solucion del sistema de ecuaciones es $x=3$ e $y=2$.
\\ \\
Bueno pero ahora, ¿como podemos generalizar la resolucion para cualquier sistema de ecuaciones?

\begin{enumerate}
	\item Tomamos una ecuacion y despejamos una incognita.
	\item Reemplazamos su equivalencia en todas las otras ecuaciones donde aparezca.
	\item Ahi obtenemos un sistema con una incognita menos.
	\item Volvemos a 1.
\end{enumerate}

Repetimos este proceso hasta obtener una ecuacion de una incognita. Al resolver esa ecuacion, vamos a tener el valor de una incognita, procederemos a reemplazar ese valor en las otras ecuaciones como hicimos en el ejemplo.
\end{normalsize}

\begin{ejemplo}
Dar la solucion del siguiente sistema de ecuaciones:
% x=3 y=7 z=5
	\begin{equation}
		\begin{cases}
		x+y+z=15\\
		x+3y=24\nonumber\\
		2x-y+4z=19
		\end{cases}
	\end{equation}
\begin{normalsize}
Paso 1: Tomemos la primer ecuacion y despejemos $x$.
\begin{align}
	x+y+z&=15\nonumber\\
	x&=15-y-z\nonumber
\end{align}
Paso 2: Reemplazamos la equivalencia de $x$ en las demas ecuaciones.
\begin{align}
	x+3y&=24\nonumber\\
	(15-y-z)+3y&=24\nonumber\\
	15-y-z+3y&=24\nonumber\\
	15-z+2y&=24\nonumber\\
	-z+2y&=9\nonumber
\end{align}
\begin{align}
	2x-y+4z&=19\nonumber\\
	2(15-y-z)-y+4z&=19\nonumber\\
	30-2y-2z-y+4z&=19\nonumber\\
	30-3y+2z&=19\nonumber\\
	-3y+2z&=19-30\nonumber\\
	-3y+2z&=-11\nonumber
\end{align}
Paso 3: Obtenemos el sistema de ecuaciones formado por estas ultimas dos ecuaciones, que no dependen de $x$.
	\begin{equation}
		\begin{cases}
		-z+2y=9\nonumber\\
		-3y+2z=-11
		\end{cases}
	\end{equation}
Paso 4: Como seguimos teniendo dos incognitas, volvemos al paso 1. Notemos que redujimos el problema a un sistema de dos ecuaciones.\\\\
Paso 1: Despejemos ahora $y$ de la primer ecuacion.
\begin{align}
	-z+2y&=9\nonumber\\
	2y&=9+z\nonumber\\
	y&=\frac{9+z}{2}\nonumber
\end{align}
Paso 2: Reemplacemos ahora la equivalencia de $y$ en la otra ecuacion.
\begin{align}
	-3y+2z&=-11\nonumber\\
	-3(\frac{9+z}{2})+2z&=-11\nonumber
\end{align}
Vemos que nos queda una ecuacion con una sola incognita, en este caso $z$. Resolvamos:
\begin{align}
	\frac{-3(9+z)}{2}+2z&=-11\nonumber\\
	\frac{-27-3z}{2}+2z&=-11\nonumber\\
	-\frac{27}{2}-\frac{3z}{2}+2z&=-11\nonumber\\
	-\frac{27}{2}-\frac{3}{2}z+2z&=-11\nonumber\\
	-\frac{3}{2}z+2z&=-11+\frac{27}{2}\nonumber\\
	\frac{1}{2}z&=-11+\frac{27}{2}\nonumber\\
	\frac{1}{2}z&=\frac{5}{2}\nonumber\\
	2(\frac{1}{2}z)&=2(\frac{5}{2})\nonumber\\
	z&=5\nonumber
\end{align}
Ahora que obtuvimos el valor de $z=5$, podemos reemplazar en alguna de las ecuaciones donde aparezca $z$. La pregunta es, ¿en cual? Porque podriamos reemplazar en la la primer ecuacion del sistema original, pero veamos que pasa si hacemos eso:
\begin{align}
	x+y+z&=15\nonumber\\
	x+y+5&=15\nonumber\\
	x+y&=10\nonumber
\end{align}
¡Nos quedo un sistema con dos ecuaciones!\\
Tendriamos que volver a resolver este sistema. Pero como los valores de $x$, $y$ y $z$ son una solucion del sistema, todas las ecuaciones intermedias son validas, es decir que podemos buscar alguna ecuacion tal que, reemplazando el valor de $z$, dependa solamente de una sola variable mas, despejamos y resolvemos.\\
Pero no hace falta ponerse a buscar ninguna ecuacion, mejor tomemos la que usamos en el paso 1 anterior, es decir $y=\frac{9+z}{2}$, porque ya tenemos la variable $y$ despejada y el valor de $z$ conocido. Por lo tanto, $y=\frac{9+5}{2}=\frac{14}{2}=7$.\\
Por ultimo, nos falta el valor de $x$. Traemos la ecuacion donde habiamos despejado $x$ al principio y nos queda, $x=15-y-z=15-7-5=3$.\\\\
La solucion del sistema de ecuaciones planteado es $x=3$, $y=7$ y $z=5$.
\end{normalsize}
\end{ejemplo}

\newpage
\subsection{Ejercicios de ecuaciones}
\begin{enumerate}
	\item La asociación de vecinos vende bonos contribución. Hay bonos de \$20 y de \$8. La cantidad de bonos de \$8 que se vendió es el triple de la cantidad de bonos de \$20 que se vendió. En total se recaudaron \$1100. ¿Cuántos bonos de cada clase se vendieron? 
	\item En la librería, Juan compró 2 carpetas y 3 marcadores; pagó en total \$327. Pablo compró 4 marcadores y pagó \$180. ¿Cuánto pagó Juan por cada carpeta?
	\item En una verdulería, cuando abrió esta mañana, el peso de todas las papas era el triple del peso de todas las batatas. Entre papas y batatas había, en total, 72 kilos. Al mediodía, antes de cerrar, quedaban 16 kilos de papas. ¿Cuántos kilos de papas se vendieron esta mañana?
	\item Luis y Pedro almuerzan los 5 días de la semana en el mismo restaurante. El plato del día tiene siempre el mismo precio y cuesta el triple que la gaseosa. El lunes, el martes y el jueves, Luis comió el plato del día con gaseosa y helado. El miércoles y el viernes, Luis sólo comió el plato del día con la gaseosa. En total, Luis gastó \$984. Pedro comió, los 5 días, el plato del día con gaseosa y helado. En total, Pedro gastó \$1080. ¿Cuánto cuesta el helado? ¿Cuánto cuesta el plato del día? ¿Cuánto cuesta la gaseosa?\\ \begin{small}\textbf{Ayuda:} Pensar en un sistema con 3 ecuaciones donde las incognitas son los precios del helado ($H$), el plato del dia ($P$) y la gaseosa ($G$).\end{small}
\end{enumerate}

\newpage
\section{Geometria}
\begin{normalsize}
Los problemas de geometria suelen ser muy concretos en lo que se pide. Tenemos una figura compuesta por varias figuras basica, nos pueden dar medidas de lados, angulos, perimetros, areas, etc.\\
Es comun que los vertices de una figura (o puntos en el dibujo) esten representados por letras para distinguir a cada uno del resto. Entonces introduzcamos notaciones que seran utilizadas en esta seccion.\\\\
\textbf{Notacion:} Para referirnos a un lado que conecta a $A$ con $B$, vamos a usar $\overline{AB}$. Entonces, si queremos decir que el lado $\overline{AB}$ mide $5cm$ ponemos $\overline{AB}=5cm$\\\\
\textbf{Notacion:} Para referirnos al angulo que forman los puntos $A$, $B$ y $C$, vamos a usar $\widehat{ABC}$. Para trabajar con angulos vamos a utilizar los \textit{grados}. $\widehat{ABC}=90$°\\\\
\end{normalsize}

\end{document}